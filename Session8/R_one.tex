\documentclass[]{article}
\usepackage{lmodern}
\usepackage{amssymb,amsmath}
\usepackage{ifxetex,ifluatex}
\usepackage{fixltx2e} % provides \textsubscript
\ifnum 0\ifxetex 1\fi\ifluatex 1\fi=0 % if pdftex
  \usepackage[T1]{fontenc}
  \usepackage[utf8]{inputenc}
\else % if luatex or xelatex
  \ifxetex
    \usepackage{mathspec}
  \else
    \usepackage{fontspec}
  \fi
  \defaultfontfeatures{Ligatures=TeX,Scale=MatchLowercase}
\fi
% use upquote if available, for straight quotes in verbatim environments
\IfFileExists{upquote.sty}{\usepackage{upquote}}{}
% use microtype if available
\IfFileExists{microtype.sty}{%
\usepackage{microtype}
\UseMicrotypeSet[protrusion]{basicmath} % disable protrusion for tt fonts
}{}
\usepackage[margin=1in]{geometry}
\usepackage{hyperref}
\hypersetup{unicode=true,
            pdftitle={Cancer Analysis},
            pdfauthor={Joshua Matthew},
            pdfborder={0 0 0},
            breaklinks=true}
\urlstyle{same}  % don't use monospace font for urls
\usepackage{color}
\usepackage{fancyvrb}
\newcommand{\VerbBar}{|}
\newcommand{\VERB}{\Verb[commandchars=\\\{\}]}
\DefineVerbatimEnvironment{Highlighting}{Verbatim}{commandchars=\\\{\}}
% Add ',fontsize=\small' for more characters per line
\usepackage{framed}
\definecolor{shadecolor}{RGB}{248,248,248}
\newenvironment{Shaded}{\begin{snugshade}}{\end{snugshade}}
\newcommand{\AlertTok}[1]{\textcolor[rgb]{0.94,0.16,0.16}{#1}}
\newcommand{\AnnotationTok}[1]{\textcolor[rgb]{0.56,0.35,0.01}{\textbf{\textit{#1}}}}
\newcommand{\AttributeTok}[1]{\textcolor[rgb]{0.77,0.63,0.00}{#1}}
\newcommand{\BaseNTok}[1]{\textcolor[rgb]{0.00,0.00,0.81}{#1}}
\newcommand{\BuiltInTok}[1]{#1}
\newcommand{\CharTok}[1]{\textcolor[rgb]{0.31,0.60,0.02}{#1}}
\newcommand{\CommentTok}[1]{\textcolor[rgb]{0.56,0.35,0.01}{\textit{#1}}}
\newcommand{\CommentVarTok}[1]{\textcolor[rgb]{0.56,0.35,0.01}{\textbf{\textit{#1}}}}
\newcommand{\ConstantTok}[1]{\textcolor[rgb]{0.00,0.00,0.00}{#1}}
\newcommand{\ControlFlowTok}[1]{\textcolor[rgb]{0.13,0.29,0.53}{\textbf{#1}}}
\newcommand{\DataTypeTok}[1]{\textcolor[rgb]{0.13,0.29,0.53}{#1}}
\newcommand{\DecValTok}[1]{\textcolor[rgb]{0.00,0.00,0.81}{#1}}
\newcommand{\DocumentationTok}[1]{\textcolor[rgb]{0.56,0.35,0.01}{\textbf{\textit{#1}}}}
\newcommand{\ErrorTok}[1]{\textcolor[rgb]{0.64,0.00,0.00}{\textbf{#1}}}
\newcommand{\ExtensionTok}[1]{#1}
\newcommand{\FloatTok}[1]{\textcolor[rgb]{0.00,0.00,0.81}{#1}}
\newcommand{\FunctionTok}[1]{\textcolor[rgb]{0.00,0.00,0.00}{#1}}
\newcommand{\ImportTok}[1]{#1}
\newcommand{\InformationTok}[1]{\textcolor[rgb]{0.56,0.35,0.01}{\textbf{\textit{#1}}}}
\newcommand{\KeywordTok}[1]{\textcolor[rgb]{0.13,0.29,0.53}{\textbf{#1}}}
\newcommand{\NormalTok}[1]{#1}
\newcommand{\OperatorTok}[1]{\textcolor[rgb]{0.81,0.36,0.00}{\textbf{#1}}}
\newcommand{\OtherTok}[1]{\textcolor[rgb]{0.56,0.35,0.01}{#1}}
\newcommand{\PreprocessorTok}[1]{\textcolor[rgb]{0.56,0.35,0.01}{\textit{#1}}}
\newcommand{\RegionMarkerTok}[1]{#1}
\newcommand{\SpecialCharTok}[1]{\textcolor[rgb]{0.00,0.00,0.00}{#1}}
\newcommand{\SpecialStringTok}[1]{\textcolor[rgb]{0.31,0.60,0.02}{#1}}
\newcommand{\StringTok}[1]{\textcolor[rgb]{0.31,0.60,0.02}{#1}}
\newcommand{\VariableTok}[1]{\textcolor[rgb]{0.00,0.00,0.00}{#1}}
\newcommand{\VerbatimStringTok}[1]{\textcolor[rgb]{0.31,0.60,0.02}{#1}}
\newcommand{\WarningTok}[1]{\textcolor[rgb]{0.56,0.35,0.01}{\textbf{\textit{#1}}}}
\usepackage{graphicx,grffile}
\makeatletter
\def\maxwidth{\ifdim\Gin@nat@width>\linewidth\linewidth\else\Gin@nat@width\fi}
\def\maxheight{\ifdim\Gin@nat@height>\textheight\textheight\else\Gin@nat@height\fi}
\makeatother
% Scale images if necessary, so that they will not overflow the page
% margins by default, and it is still possible to overwrite the defaults
% using explicit options in \includegraphics[width, height, ...]{}
\setkeys{Gin}{width=\maxwidth,height=\maxheight,keepaspectratio}
\IfFileExists{parskip.sty}{%
\usepackage{parskip}
}{% else
\setlength{\parindent}{0pt}
\setlength{\parskip}{6pt plus 2pt minus 1pt}
}
\setlength{\emergencystretch}{3em}  % prevent overfull lines
\providecommand{\tightlist}{%
  \setlength{\itemsep}{0pt}\setlength{\parskip}{0pt}}
\setcounter{secnumdepth}{0}
% Redefines (sub)paragraphs to behave more like sections
\ifx\paragraph\undefined\else
\let\oldparagraph\paragraph
\renewcommand{\paragraph}[1]{\oldparagraph{#1}\mbox{}}
\fi
\ifx\subparagraph\undefined\else
\let\oldsubparagraph\subparagraph
\renewcommand{\subparagraph}[1]{\oldsubparagraph{#1}\mbox{}}
\fi

%%% Use protect on footnotes to avoid problems with footnotes in titles
\let\rmarkdownfootnote\footnote%
\def\footnote{\protect\rmarkdownfootnote}

%%% Change title format to be more compact
\usepackage{titling}

% Create subtitle command for use in maketitle
\providecommand{\subtitle}[1]{
  \posttitle{
    \begin{center}\large#1\end{center}
    }
}

\setlength{\droptitle}{-2em}

  \title{Cancer Analysis}
    \pretitle{\vspace{\droptitle}\centering\huge}
  \posttitle{\par}
    \author{Joshua Matthew}
    \preauthor{\centering\large\emph}
  \postauthor{\par}
      \predate{\centering\large\emph}
  \postdate{\par}
    \date{5/12/2019}


\begin{document}
\maketitle

\begin{Shaded}
\begin{Highlighting}[]
\KeywordTok{library}\NormalTok{(caret)}
\end{Highlighting}
\end{Shaded}

\begin{verbatim}
## Loading required package: lattice
\end{verbatim}

\begin{verbatim}
## Loading required package: ggplot2
\end{verbatim}

\begin{Shaded}
\begin{Highlighting}[]
\KeywordTok{library}\NormalTok{(rsample)}
\end{Highlighting}
\end{Shaded}

\begin{verbatim}
## Loading required package: tidyr
\end{verbatim}

\begin{Shaded}
\begin{Highlighting}[]
\KeywordTok{library}\NormalTok{(tidyverse)}
\end{Highlighting}
\end{Shaded}

\begin{verbatim}
## -- Attaching packages -------------------------------------- tidyverse 1.2.1 --
\end{verbatim}

\begin{verbatim}
## v tibble  2.1.3     v dplyr   0.8.3
## v readr   1.3.1     v stringr 1.4.0
## v purrr   0.3.2     v forcats 0.4.0
\end{verbatim}

\begin{verbatim}
## -- Conflicts ----------------------------------------- tidyverse_conflicts() --
## x dplyr::filter() masks stats::filter()
## x dplyr::lag()    masks stats::lag()
## x purrr::lift()   masks caret::lift()
\end{verbatim}

\begin{Shaded}
\begin{Highlighting}[]
\KeywordTok{library}\NormalTok{(pROC)}
\end{Highlighting}
\end{Shaded}

\begin{verbatim}
## Type 'citation("pROC")' for a citation.
\end{verbatim}

\begin{verbatim}
## 
## Attaching package: 'pROC'
\end{verbatim}

\begin{verbatim}
## The following objects are masked from 'package:stats':
## 
##     cov, smooth, var
\end{verbatim}

\begin{Shaded}
\begin{Highlighting}[]
\NormalTok{cnc<-}\KeywordTok{read.csv}\NormalTok{(}\StringTok{"data/cancer.csv"}\NormalTok{, }\DataTypeTok{header=}\NormalTok{T)}
\end{Highlighting}
\end{Shaded}

\begin{Shaded}
\begin{Highlighting}[]
\KeywordTok{set.seed}\NormalTok{(}\DecValTok{100}\NormalTok{)}
\NormalTok{train_test_split <-}\StringTok{ }\KeywordTok{initial_split}\NormalTok{(cnc, }\DataTypeTok{prop =} \FloatTok{0.8}\NormalTok{)}
\NormalTok{train_test_split}
\end{Highlighting}
\end{Shaded}

\begin{verbatim}
## <456/113/569>
\end{verbatim}

\begin{Shaded}
\begin{Highlighting}[]
\NormalTok{train_set <-}\StringTok{ }\KeywordTok{training}\NormalTok{(train_test_split)}
\NormalTok{test_set <-}\StringTok{ }\KeywordTok{testing}\NormalTok{(train_test_split)}
\end{Highlighting}
\end{Shaded}

\begin{Shaded}
\begin{Highlighting}[]
\KeywordTok{sapply}\NormalTok{(cnc, }\ControlFlowTok{function}\NormalTok{(x) }\KeywordTok{sum}\NormalTok{(}\KeywordTok{is.na}\NormalTok{(x)))}
\end{Highlighting}
\end{Shaded}

\begin{verbatim}
##                      id               diagnosis             radius_mean 
##                       0                       0                       0 
##            texture_mean          perimeter_mean               area_mean 
##                       0                       0                       0 
##         smoothness_mean        compactness_mean          concavity_mean 
##                       0                       0                       0 
##     concave.points_mean           symmetry_mean  fractal_dimension_mean 
##                       0                       0                       0 
##               radius_se              texture_se            perimeter_se 
##                       0                       0                       0 
##                 area_se           smoothness_se          compactness_se 
##                       0                       0                       0 
##            concavity_se       concave.points_se             symmetry_se 
##                       0                       0                       0 
##    fractal_dimension_se            radius_worst           texture_worst 
##                       0                       0                       0 
##         perimeter_worst              area_worst        smoothness_worst 
##                       0                       0                       0 
##       compactness_worst         concavity_worst    concave.points_worst 
##                       0                       0                       0 
##          symmetry_worst fractal_dimension_worst 
##                       0                       0
\end{verbatim}

\begin{Shaded}
\begin{Highlighting}[]
\NormalTok{train_set}\OperatorTok{$}\NormalTok{id<-}\OtherTok{NULL}
\NormalTok{test_set}\OperatorTok{$}\NormalTok{id<-}\OtherTok{NULL}
\end{Highlighting}
\end{Shaded}

\begin{Shaded}
\begin{Highlighting}[]
\NormalTok{control <-}\StringTok{ }\KeywordTok{trainControl}\NormalTok{(}\DataTypeTok{method=}\StringTok{"repeatedcv"}\NormalTok{, }\DataTypeTok{number=}\DecValTok{10}\NormalTok{, }\DataTypeTok{repeats=}\DecValTok{3}\NormalTok{)}
\NormalTok{model <-}\StringTok{ }\KeywordTok{train}\NormalTok{(diagnosis}\OperatorTok{~}\NormalTok{., }\DataTypeTok{data=}\NormalTok{train_set, }\DataTypeTok{method=}\StringTok{"lvq"}\NormalTok{, }\DataTypeTok{preProcess=}\StringTok{"scale"}\NormalTok{, }\DataTypeTok{trControl=}\NormalTok{control)}
\NormalTok{importance <-}\StringTok{ }\KeywordTok{varImp}\NormalTok{(model, }\DataTypeTok{scale=}\OtherTok{FALSE}\NormalTok{)}
\KeywordTok{plot}\NormalTok{(importance)}
\end{Highlighting}
\end{Shaded}

\includegraphics{R_one_files/figure-latex/unnamed-chunk-7-1.pdf}

\begin{Shaded}
\begin{Highlighting}[]
\NormalTok{train_new<-train_set[,}\OperatorTok{-}\KeywordTok{c}\NormalTok{(}\DecValTok{13}\NormalTok{,}\DecValTok{16}\NormalTok{,}\DecValTok{20}\NormalTok{,}\DecValTok{21}\NormalTok{)]}
\NormalTok{test_new<-test_set[,}\OperatorTok{-}\KeywordTok{c}\NormalTok{(}\DecValTok{13}\NormalTok{,}\DecValTok{16}\NormalTok{,}\DecValTok{20}\NormalTok{,}\DecValTok{21}\NormalTok{)]}
\end{Highlighting}
\end{Shaded}

\begin{Shaded}
\begin{Highlighting}[]
\NormalTok{control <-}\StringTok{ }\KeywordTok{trainControl}\NormalTok{(}\StringTok{"repeatedcv"}\NormalTok{, }\DataTypeTok{number =} \DecValTok{10}\NormalTok{, }\DataTypeTok{repeats =} \DecValTok{3}\NormalTok{)}
\end{Highlighting}
\end{Shaded}

Logistic Regression

\begin{Shaded}
\begin{Highlighting}[]
\NormalTok{logis<-}\StringTok{ }\KeywordTok{train}\NormalTok{(}\DataTypeTok{form=}\NormalTok{diagnosis}\OperatorTok{~}\NormalTok{., }\DataTypeTok{data=}\NormalTok{train_new,}\DataTypeTok{method=}\StringTok{"glm"}\NormalTok{, }\DataTypeTok{family=}\StringTok{"binomial"}\NormalTok{, }\DataTypeTok{preProcess =} \KeywordTok{c}\NormalTok{(}\StringTok{"center"}\NormalTok{, }\StringTok{"scale"}\NormalTok{), }\DataTypeTok{trControl=}\NormalTok{control,}\DataTypeTok{tuneLength =} \DecValTok{5}\NormalTok{)}
\end{Highlighting}
\end{Shaded}

\begin{verbatim}
## Warning: glm.fit: algorithm did not converge
\end{verbatim}

\begin{verbatim}
## Warning: glm.fit: fitted probabilities numerically 0 or 1 occurred
\end{verbatim}

\begin{verbatim}
## Warning: glm.fit: algorithm did not converge
\end{verbatim}

\begin{verbatim}
## Warning: glm.fit: fitted probabilities numerically 0 or 1 occurred
\end{verbatim}

\begin{verbatim}
## Warning: glm.fit: algorithm did not converge
\end{verbatim}

\begin{verbatim}
## Warning: glm.fit: fitted probabilities numerically 0 or 1 occurred
\end{verbatim}

\begin{verbatim}
## Warning: glm.fit: algorithm did not converge
\end{verbatim}

\begin{verbatim}
## Warning: glm.fit: fitted probabilities numerically 0 or 1 occurred
\end{verbatim}

\begin{verbatim}
## Warning: glm.fit: algorithm did not converge
\end{verbatim}

\begin{verbatim}
## Warning: glm.fit: fitted probabilities numerically 0 or 1 occurred
\end{verbatim}

\begin{verbatim}
## Warning: glm.fit: algorithm did not converge
\end{verbatim}

\begin{verbatim}
## Warning: glm.fit: fitted probabilities numerically 0 or 1 occurred
\end{verbatim}

\begin{verbatim}
## Warning: glm.fit: algorithm did not converge
\end{verbatim}

\begin{verbatim}
## Warning: glm.fit: fitted probabilities numerically 0 or 1 occurred
\end{verbatim}

\begin{verbatim}
## Warning: glm.fit: algorithm did not converge
\end{verbatim}

\begin{verbatim}
## Warning: glm.fit: fitted probabilities numerically 0 or 1 occurred
\end{verbatim}

\begin{verbatim}
## Warning: glm.fit: algorithm did not converge
\end{verbatim}

\begin{verbatim}
## Warning: glm.fit: fitted probabilities numerically 0 or 1 occurred
\end{verbatim}

\begin{verbatim}
## Warning: glm.fit: algorithm did not converge
\end{verbatim}

\begin{verbatim}
## Warning: glm.fit: fitted probabilities numerically 0 or 1 occurred
\end{verbatim}

\begin{verbatim}
## Warning: glm.fit: algorithm did not converge
\end{verbatim}

\begin{verbatim}
## Warning: glm.fit: fitted probabilities numerically 0 or 1 occurred
\end{verbatim}

\begin{verbatim}
## Warning: glm.fit: algorithm did not converge
\end{verbatim}

\begin{verbatim}
## Warning: glm.fit: fitted probabilities numerically 0 or 1 occurred
\end{verbatim}

\begin{verbatim}
## Warning: glm.fit: algorithm did not converge
\end{verbatim}

\begin{verbatim}
## Warning: glm.fit: fitted probabilities numerically 0 or 1 occurred
\end{verbatim}

\begin{verbatim}
## Warning: glm.fit: algorithm did not converge
\end{verbatim}

\begin{verbatim}
## Warning: glm.fit: fitted probabilities numerically 0 or 1 occurred
\end{verbatim}

\begin{verbatim}
## Warning: glm.fit: algorithm did not converge
\end{verbatim}

\begin{verbatim}
## Warning: glm.fit: fitted probabilities numerically 0 or 1 occurred
\end{verbatim}

\begin{verbatim}
## Warning: glm.fit: algorithm did not converge
\end{verbatim}

\begin{verbatim}
## Warning: glm.fit: fitted probabilities numerically 0 or 1 occurred
\end{verbatim}

\begin{verbatim}
## Warning: glm.fit: algorithm did not converge
\end{verbatim}

\begin{verbatim}
## Warning: glm.fit: fitted probabilities numerically 0 or 1 occurred
\end{verbatim}

\begin{verbatim}
## Warning: glm.fit: algorithm did not converge
\end{verbatim}

\begin{verbatim}
## Warning: glm.fit: fitted probabilities numerically 0 or 1 occurred
\end{verbatim}

\begin{verbatim}
## Warning: glm.fit: algorithm did not converge
\end{verbatim}

\begin{verbatim}
## Warning: glm.fit: fitted probabilities numerically 0 or 1 occurred
\end{verbatim}

\begin{verbatim}
## Warning: glm.fit: algorithm did not converge
\end{verbatim}

\begin{verbatim}
## Warning: glm.fit: fitted probabilities numerically 0 or 1 occurred
\end{verbatim}

\begin{verbatim}
## Warning: glm.fit: algorithm did not converge
\end{verbatim}

\begin{verbatim}
## Warning: glm.fit: fitted probabilities numerically 0 or 1 occurred
\end{verbatim}

\begin{verbatim}
## Warning: glm.fit: algorithm did not converge
\end{verbatim}

\begin{verbatim}
## Warning: glm.fit: fitted probabilities numerically 0 or 1 occurred
\end{verbatim}

\begin{verbatim}
## Warning: glm.fit: algorithm did not converge
\end{verbatim}

\begin{verbatim}
## Warning: glm.fit: fitted probabilities numerically 0 or 1 occurred
\end{verbatim}

\begin{verbatim}
## Warning: glm.fit: algorithm did not converge
\end{verbatim}

\begin{verbatim}
## Warning: glm.fit: fitted probabilities numerically 0 or 1 occurred
\end{verbatim}

\begin{verbatim}
## Warning: glm.fit: algorithm did not converge
\end{verbatim}

\begin{verbatim}
## Warning: glm.fit: fitted probabilities numerically 0 or 1 occurred
\end{verbatim}

\begin{verbatim}
## Warning: glm.fit: algorithm did not converge
\end{verbatim}

\begin{verbatim}
## Warning: glm.fit: fitted probabilities numerically 0 or 1 occurred
\end{verbatim}

\begin{verbatim}
## Warning: glm.fit: algorithm did not converge
\end{verbatim}

\begin{verbatim}
## Warning: glm.fit: fitted probabilities numerically 0 or 1 occurred
\end{verbatim}

\begin{verbatim}
## Warning: glm.fit: algorithm did not converge
\end{verbatim}

\begin{verbatim}
## Warning: glm.fit: fitted probabilities numerically 0 or 1 occurred
\end{verbatim}

\begin{verbatim}
## Warning: glm.fit: algorithm did not converge
\end{verbatim}

\begin{verbatim}
## Warning: glm.fit: fitted probabilities numerically 0 or 1 occurred
\end{verbatim}

\begin{verbatim}
## Warning: glm.fit: algorithm did not converge
\end{verbatim}

\begin{verbatim}
## Warning: glm.fit: fitted probabilities numerically 0 or 1 occurred
\end{verbatim}

\begin{verbatim}
## Warning: glm.fit: algorithm did not converge
\end{verbatim}

\begin{verbatim}
## Warning: glm.fit: fitted probabilities numerically 0 or 1 occurred
\end{verbatim}

Surface Vector Machines

\begin{Shaded}
\begin{Highlighting}[]
\NormalTok{svm<-}\StringTok{ }\KeywordTok{train}\NormalTok{(}\DataTypeTok{form=}\NormalTok{diagnosis}\OperatorTok{~}\NormalTok{., }\DataTypeTok{data=}\NormalTok{train_new, }\DataTypeTok{method=}\StringTok{"svmLinear"}\NormalTok{, }\DataTypeTok{preProcess =} \KeywordTok{c}\NormalTok{(}\StringTok{"center"}\NormalTok{, }\StringTok{"scale"}\NormalTok{), }\DataTypeTok{trControl=}\NormalTok{control, }\DataTypeTok{tuneLength =}\DecValTok{5}\NormalTok{)}
\end{Highlighting}
\end{Shaded}

Decision Trees

\begin{Shaded}
\begin{Highlighting}[]
\NormalTok{dct<-}\StringTok{ }\KeywordTok{train}\NormalTok{(}\DataTypeTok{form=}\NormalTok{diagnosis}\OperatorTok{~}\NormalTok{., }\DataTypeTok{data=}\NormalTok{train_new,}\DataTypeTok{method=}\StringTok{"rpart"}\NormalTok{, }\DataTypeTok{metric=}\StringTok{"Accuracy"}\NormalTok{,}\DataTypeTok{preProcess =} \KeywordTok{c}\NormalTok{(}\StringTok{"center"}\NormalTok{, }\StringTok{"scale"}\NormalTok{),}\DataTypeTok{trControl=}\NormalTok{control, }\DataTypeTok{tuneLength =} \DecValTok{5}\NormalTok{)}
\end{Highlighting}
\end{Shaded}

\hypertarget{predicting-svm}{%
\subsection{Predicting SVM}\label{predicting-svm}}

\begin{Shaded}
\begin{Highlighting}[]
\NormalTok{predsvm<-}\KeywordTok{predict}\NormalTok{(svm,test_new,}\DataTypeTok{type=}\StringTok{"raw"}\NormalTok{)}
\KeywordTok{table}\NormalTok{(predsvm, test_new}\OperatorTok{$}\NormalTok{diagnosis)}
\end{Highlighting}
\end{Shaded}

\begin{verbatim}
##        
## predsvm  B  M
##       B 71  5
##       M  1 36
\end{verbatim}

\hypertarget{predicting-logistic}{%
\subsection{Predicting Logistic}\label{predicting-logistic}}

\begin{Shaded}
\begin{Highlighting}[]
\NormalTok{predlog<-}\KeywordTok{predict}\NormalTok{(logis,test_new,}\DataTypeTok{type=}\StringTok{"raw"}\NormalTok{)}
\KeywordTok{table}\NormalTok{(predlog, test_new}\OperatorTok{$}\NormalTok{diagnosis)}
\end{Highlighting}
\end{Shaded}

\begin{verbatim}
##        
## predlog  B  M
##       B 67  4
##       M  5 37
\end{verbatim}

\hypertarget{predicting-decision-tree}{%
\subsection{Predicting Decision Tree}\label{predicting-decision-tree}}

\begin{Shaded}
\begin{Highlighting}[]
\NormalTok{preddct<-}\KeywordTok{predict}\NormalTok{(dct,test_new,}\DataTypeTok{type=}\StringTok{"raw"}\NormalTok{)}
\KeywordTok{table}\NormalTok{(preddct, test_new}\OperatorTok{$}\NormalTok{diagnosis)}
\end{Highlighting}
\end{Shaded}

\begin{verbatim}
##        
## preddct  B  M
##       B 66  6
##       M  6 35
\end{verbatim}

\hypertarget{roc-curve-for-surface-vector-machine}{%
\subsection{ROC Curve for Surface Vector
Machine}\label{roc-curve-for-surface-vector-machine}}

\begin{Shaded}
\begin{Highlighting}[]
\NormalTok{response1 <-}\StringTok{ }\NormalTok{predictor1 <-}\StringTok{ }\KeywordTok{c}\NormalTok{()}
\NormalTok{response1 <-}\StringTok{ }\KeywordTok{c}\NormalTok{(response1, test_new}\OperatorTok{$}\NormalTok{diagnosis)}
\NormalTok{predictor1<-}\StringTok{ }\KeywordTok{c}\NormalTok{(predictor1,predsvm) }


\NormalTok{roc1 <-}\StringTok{ }\KeywordTok{plot.roc}\NormalTok{(response1, predictor1,  }\DataTypeTok{main=}\StringTok{"ROC for SVM"}\NormalTok{,}
\DataTypeTok{ylab=}\StringTok{"True Positive Rate"}\NormalTok{,}\DataTypeTok{xlab=}\StringTok{"False Positive Rate"}\NormalTok{, }\DataTypeTok{percent=}\OtherTok{TRUE}\NormalTok{, }\DataTypeTok{col=}\StringTok{"green"}\NormalTok{) }
\end{Highlighting}
\end{Shaded}

\begin{verbatim}
## Setting levels: control = 1, case = 2
\end{verbatim}

\begin{verbatim}
## Setting direction: controls < cases
\end{verbatim}

\includegraphics{R_one_files/figure-latex/unnamed-chunk-16-1.pdf}

\hypertarget{roc-curve-for-decision-tree}{%
\subsection{ROC Curve for Decision
Tree}\label{roc-curve-for-decision-tree}}

\begin{Shaded}
\begin{Highlighting}[]
\NormalTok{response2 <-}\StringTok{ }\NormalTok{predictor2 <-}\StringTok{ }\KeywordTok{c}\NormalTok{()}
\NormalTok{response2 <-}\StringTok{ }\KeywordTok{c}\NormalTok{(response2, test_new}\OperatorTok{$}\NormalTok{diagnosis)}
\NormalTok{predictor2 <-}\StringTok{ }\KeywordTok{c}\NormalTok{(predictor2, preddct)}

\NormalTok{roc2 <-}\StringTok{ }\KeywordTok{plot.roc}\NormalTok{(response2, predictor2,  }\DataTypeTok{main=}\StringTok{"ROC for DT"}\NormalTok{,}
                 \DataTypeTok{ylab=}\StringTok{"True Positive Rate"}\NormalTok{,}\DataTypeTok{xlab=}\StringTok{"False Positive Rate"}\NormalTok{, }\DataTypeTok{percent=}\OtherTok{TRUE}\NormalTok{, }\DataTypeTok{col=}\StringTok{"black"}\NormalTok{)}
\end{Highlighting}
\end{Shaded}

\begin{verbatim}
## Setting levels: control = 1, case = 2
\end{verbatim}

\begin{verbatim}
## Setting direction: controls < cases
\end{verbatim}

\includegraphics{R_one_files/figure-latex/unnamed-chunk-17-1.pdf}

\hypertarget{roc-curve-for-logistic-regression}{%
\subsection{ROC Curve for Logistic
Regression}\label{roc-curve-for-logistic-regression}}

\begin{Shaded}
\begin{Highlighting}[]
\NormalTok{response4 <-}\StringTok{ }\NormalTok{predictor4 <-}\StringTok{ }\KeywordTok{c}\NormalTok{()}
\NormalTok{response4 <-}\StringTok{ }\KeywordTok{c}\NormalTok{(response4, test_new}\OperatorTok{$}\NormalTok{diagnosis)}
\NormalTok{predictor4 <-}\StringTok{ }\KeywordTok{c}\NormalTok{(predictor4, predlog)}

\NormalTok{roc4<-}\StringTok{ }\KeywordTok{plot.roc}\NormalTok{(response4, predictor4,  }\DataTypeTok{main=}\StringTok{"ROC for LR"}\NormalTok{,}
                \DataTypeTok{ylab=}\StringTok{"True Positive Rate"}\NormalTok{,}\DataTypeTok{xlab=}\StringTok{"False Positive Rate"}\NormalTok{, }\DataTypeTok{percent=}\OtherTok{TRUE}\NormalTok{, }\DataTypeTok{col=}\StringTok{"magenta"}\NormalTok{)}
\end{Highlighting}
\end{Shaded}

\begin{verbatim}
## Setting levels: control = 1, case = 2
\end{verbatim}

\begin{verbatim}
## Setting direction: controls < cases
\end{verbatim}

\includegraphics{R_one_files/figure-latex/unnamed-chunk-18-1.pdf}

All ROC

\begin{Shaded}
\begin{Highlighting}[]
\NormalTok{roc1 <-}\StringTok{ }\KeywordTok{plot.roc}\NormalTok{(response1, predictor1,  }\DataTypeTok{main=}\StringTok{"ROC for SVM, LR and DT"}\NormalTok{,}
\DataTypeTok{ylab=}\StringTok{"True Positive Rate"}\NormalTok{,}\DataTypeTok{xlab=}\StringTok{"False Positive Rate"}\NormalTok{, }\DataTypeTok{percent=}\OtherTok{TRUE}\NormalTok{, }\DataTypeTok{col=}\StringTok{"green"}\NormalTok{) }
\end{Highlighting}
\end{Shaded}

\begin{verbatim}
## Setting levels: control = 1, case = 2
\end{verbatim}

\begin{verbatim}
## Setting direction: controls < cases
\end{verbatim}

\begin{Shaded}
\begin{Highlighting}[]
\KeywordTok{par}\NormalTok{(}\DataTypeTok{new=}\OtherTok{TRUE}\NormalTok{)}
\NormalTok{roc2 <-}\StringTok{ }\KeywordTok{plot.roc}\NormalTok{(response2, predictor2,  }\DataTypeTok{main=}\StringTok{"ROC for SVM, LR and DT"}\NormalTok{,}
                 \DataTypeTok{ylab=}\StringTok{"True Positive Rate"}\NormalTok{,}\DataTypeTok{xlab=}\StringTok{"False Positive Rate"}\NormalTok{, }\DataTypeTok{percent=}\OtherTok{TRUE}\NormalTok{, }\DataTypeTok{col=}\StringTok{"black"}\NormalTok{)}
\end{Highlighting}
\end{Shaded}

\begin{verbatim}
## Setting levels: control = 1, case = 2
## Setting direction: controls < cases
\end{verbatim}

\begin{Shaded}
\begin{Highlighting}[]
\KeywordTok{par}\NormalTok{(}\DataTypeTok{new=}\OtherTok{TRUE}\NormalTok{)}
\NormalTok{roc4<-}\StringTok{ }\KeywordTok{plot.roc}\NormalTok{(response4, predictor4,  }\DataTypeTok{main=}\StringTok{"ROC for SVM, LR and DT"}\NormalTok{,}
                \DataTypeTok{ylab=}\StringTok{"True Positive Rate"}\NormalTok{,}\DataTypeTok{xlab=}\StringTok{"False Positive Rate"}\NormalTok{, }\DataTypeTok{percent=}\OtherTok{TRUE}\NormalTok{, }\DataTypeTok{col=}\StringTok{"magenta"}\NormalTok{)}
\end{Highlighting}
\end{Shaded}

\begin{verbatim}
## Setting levels: control = 1, case = 2
## Setting direction: controls < cases
\end{verbatim}

\begin{Shaded}
\begin{Highlighting}[]
\KeywordTok{legend}\NormalTok{(}\StringTok{"bottomright"}\NormalTok{, }\DataTypeTok{legend =} \KeywordTok{c}\NormalTok{(}\StringTok{"SVM"}\NormalTok{, }\StringTok{"LR"}\NormalTok{,}\StringTok{'DT'}\NormalTok{), }\DataTypeTok{col =} \KeywordTok{c}\NormalTok{(}\StringTok{"green"}\NormalTok{, }\StringTok{"black"}\NormalTok{, }\StringTok{"magenta"}\NormalTok{),}\DataTypeTok{lwd =} \DecValTok{2}\NormalTok{)}
\end{Highlighting}
\end{Shaded}

\includegraphics{R_one_files/figure-latex/unnamed-chunk-19-1.pdf}


\end{document}
